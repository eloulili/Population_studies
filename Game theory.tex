\documentclass[11pt, a4paper]{article}
\usepackage[utf8]{inputenc}
\usepackage[T1]{fontenc}
\usepackage[francais]{babel}
\usepackage{textcomp}
\usepackage{mathtools,amssymb,amsthm}
\usepackage{lmodern} 
\usepackage{stmaryrd}

\usepackage{geometry}
\usepackage{xcolor}
\usepackage{array}
\usepackage{calc}
\usepackage{amsmath}

\usepackage{titlesec}
\usepackage{titletoc}
\usepackage{fancyhdr}
\usepackage{titling}
\usepackage{enumitem}
\usepackage{eurosym}

\usepackage{hyperref}
\hypersetup{
    colorlinks=true,
    linkcolor=blue,
    urlcolor=blue,
    filecolor=blue}
    
\geometry{hmargin=2.cm, vmargin = 2.cm}

\usepackage{diagbox} % tableaux à double entrée

\title{Algorithmic game theories : Solutions Concept}
\author{Eliel Dzik}

\begin{document}
\date{}
\maketitle

\section{\textit{And-or} games}
(a) Let first consider that n is even. Then "min" needs to choose at steps $0 \leq 2i \leq n-1$ and therefore has $ n/2$ steps to play. At each one, there are $2^i$ scenarii depending on "max" actions, so $ 2^{2^i}$ possible actions.
In the end, "min" has $2^{\sum_{i=0}^{\frac{n-2}{2}} 2^i} = 2^{2^{\frac{n}{2} -1 }}$ different strategies.


For the same reasons, "max" has $2^{\sum_{i=1}^{\frac{n}{2}} 2^i} = 2^{2^{\frac{n}{2} + 1 } - 2 }$ different strategies


\textit{Mutatis mutandis}, if n is odd,  for \textit{max} there are $2^{2^{\frac{n-1}{2} - 1 } - 2 }$ strategies, and $ 2^{2^{\frac{n-1}{2} + 1 }-1}$ for \textit{min}.

Then the game on the normal form has the correspondent number of row and columns, and the payoff is the payoff given by the path followed in the tree.
\newline

(b) \textbf{Initial Case n = 1}: This case is trivial : if there is a leaf with -1, the players "min" chooses this one.

\textbf{Case n = 2} : if every branch has at least one +1, the strategy of max is to choose the one accessible depending on "min" 's choice. Else, "min"  chooses the branch without +1 leaves. Each strategy wins without depending on the strategy of the opponent.

\textbf{General case} : Let consider that the property is true for trees of depth $\{1, ..., n\}$. At the root, "min" can choose between two subtrees, where one of the players has a wining  strategy (hypothesis). If there is one where he wins, he chooses this one and follows the strategy. If not, then max has a winning strategy for every subtree, not depending on "min" choice.
\newline

(c) For n = 1, "max" needs two  +1 since "min" chooses.
For n = 2, max needs at least two +1 : one at each subtree and then he will choose this one at the second step.

In the general case, we need to notice that the payoff of player Max is the max of the payoffs of the subtrees. 
For Min it is the min. And we start from the leaves to deduce the value of each node. 
Then, if n is odd, Max needs at least $2^{\frac{n-1}{2}}$ +1 to win, and if n is even, he needs $2^{\frac{n}{2}}$ +1 to win.


\section{Second price auction}

(a) Both players have the same strategy : bid the value
 of the object. Let consider $a \leq b$ and denote $A,  B$ bids of each player. It is trivial that no one will bid more than the value of the object, 
 since there would be a risk of negative payoff instead of 0. For player 2, if he bids less than $a$, he will lose the auction, and if he bids more than $a$, there is no difference between $a + \epsilon $ and $b$ 
 since he will pay $a$ in both cases. For player 1, there is no strategy in which he wins more than 0. 
 If $a \geq b$, the same reasoning applies.
\newline

(b) No, it is impossible : if player 1 bids $a$, then  player 2 can bid $b$ and win the auction with a payoff of $b - a > 0$.
There is no subgame perfect equilibrium in pure strategies where  $ a \geq b$ and player 2 wins the item : if player 1 bids less than
    $b$, he will win the auction and pay $0$ instead of $b - a$, then he would have a better payoff by bidding $b$.


\section{Nash equilibrium}
Because of the symetry of the game, we can consider that row players chooses a strategy on the two first rows and the result will be 
easily extended to the other rows. We can notice that the best answer of the column player to row 1 is column 2 and, to row 2, column 3. Moreover,
 column 2 strictly dominates column 1.
Besides, there is no Nash equilibrium in pure strategy.


The mixed strategy of the row player is : $p = (p_1, 1 - p_1, 0)$.
Let denote $q = (q_1, q_2, q_3)$ the mixed strategy of the column player with $q_1 + q_2 + q_3 = 1$. Thanks to the previous
 remark, we can consider that the column player will play a mix of column 2 and 3 the strategy is then $q = (0, q_2, 1-q_2)$. The expected payoff of the row player is then :
$$
q_2(2 p_1 + 1 (1-p_1)) + 2(1-q_2) p_1 = 2q_2 p_1 + q_2 - q_2 p_1  +2p_1 - 2q_2 p_1 = q_2 + 2p_1 - q_2 p_1
$$

Then, since $ p_1 \geq 1$, the best answer is $q_2 = 1$ which is a pure strategy 
But if the column player plays pure strategies, the row player knows the best answer which is also a pure strategy. Then it means that $ p_1 =0 $ which is impossible since  there is no Nash equilibrium in pure strategies. Then ther is no Nash equilibrium 
where the row player plays a mixed strategy with only two possible rows, 




