\documentclass[11pt, a4paper]{article}
\usepackage[utf8]{inputenc}
\usepackage[T1]{fontenc}
\usepackage[francais]{babel}
\usepackage{textcomp}
\usepackage{mathtools,amssymb,amsthm}
\usepackage{lmodern} 
\usepackage{stmaryrd}

\usepackage{geometry}
\usepackage{xcolor}
\usepackage{array}
\usepackage{calc}
\usepackage{amsmath}

\usepackage{titlesec}
\usepackage{titletoc}
\usepackage{fancyhdr}
\usepackage{titling}
\usepackage{enumitem}
\usepackage{eurosym}

\usepackage{hyperref}
\hypersetup{
    colorlinks=true,
    linkcolor=blue,
    urlcolor=blue,
    filecolor=blue}
    
\geometry{hmargin=2.cm, vmargin = 2.cm}

\usepackage{diagbox} % tableaux à double entrée

\title{Algorithmic game theories : Solutions Concept}
\author{Eliel Dzik}

\begin{document}
\date{}
\maketitle

\section{Stable matching}
(a) Let consider the following instance, where men are represented by the letters $a, b, c$ and wemen by the numbers $1, 2, 3$ :
$$
a : 1 > 2 > 3, \quad b : 1 > 3 > 2, \quad c : 2 > 3 > 1
1 : a > b > c, \quad 2 : a > b > c, \quad 3 : a > b > c
$$

In this case, the Gale-Shapley algorithm gives the following matching : $a \rightarrow 1, b \rightarrow 3, c \rightarrow 2$. 

If woman 2 declares instead $c > b > a$, the matching will be $a \rightarrow 1, b \rightarrow 3, c \rightarrow 2$, which is better according to its real preferences.
However, this matching is not stable.

(b) Firstly, we can notice that since the game is finite, there is a Nash equilibrium. Now let's prove that a pure Nash equilibrium exists.

Let's suppose that there is no pure Nash equilibrium. Then, for each preference list, there is a player who can improve his matching by changing its preference list.


\section{Surprise examination paradox}
(a) If the game reaches day 4,  whatever the teacher does, the student only has to play S to win. 
Then a mixed strategy which includes to play day 4 allows a better prediction to the student and then cannot be a minimax strategy.
The randomized strategy which consists in playing a day between 1 and 3 with probability $\frac{1}{3}$ provides an expected payoff of $\frac{2}{3}$.
If the randomization is not as above, the student can always play the day which is the most probable (with probability $ p > \frac{1}{3}$ )
 to be the day of the exam and the expected payoff of the teacher will be lower than $\frac{2}{3}$. Finaly,the minimax strategy of the
 teacher is ($\frac{1}{3}, \frac{1}{3}, \frac{1}{3}, 0 $).

 \textbf{The minimax strategy of the student is} ($\frac{1}{4}, \frac{1}{4}, \frac{1}{4}, \frac{1}{4} $) with an expected payoff $-\frac{3}{4}$: if he excludes
the last day, the teacher can always play this day and the student will lose with probability 1.
If a day is played with probability $ p > \frac{1}{4}$, the teacher can randomize on the other days and the student will get a payoff lower than $-\frac{3}{4}$.


(b) In the unbounded version, \textbf{the teacher does not have a minimax strategy}. Firstly, let's notice that no strategy can ensure a payoff of 1.
Then, if there were a minimax strategy, the payoff would be lower than 1.
Let's consider that a minimax strategy exists and let's denote $\mu < 1$ its payoff. Let's $n$ such that $\frac{n-1}{n}> \mu$. Then randomizing 
on the days $1, 2, ..., n$ with probability $\frac{1}{n}$ provides a payoff of $\frac{n-1}{n} > \mu$ which is a contradiction.


No strategy of the student on a finite number of days can ensure a payoff of more than -1. 
Then, the teacher can always play the day $n+1$ and the student will lose with probability 1.
If there is a minimax strategy, the payoff would be higher than -1 and with an infinite number of days.
In expectancy, whatever the student's stategy, there is always a strategy of the teacher which provides a payoff of -1 to the student in expectancy.
Then, every strategy of the student is a minimax strategy since it cannot ensure a payoff of more than -1.