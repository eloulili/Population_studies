\documentclass[11pt, a4paper]{article}
\usepackage[utf8]{inputenc}
\usepackage[T1]{fontenc}
\usepackage[francais]{babel}
\usepackage{textcomp}
\usepackage{mathtools,amssymb,amsthm}
\usepackage{lmodern} 
\usepackage{stmaryrd}

\usepackage{geometry}
\usepackage{xcolor}
\usepackage{array}
\usepackage{calc}
\usepackage{amsmath}

\usepackage{titlesec}
\usepackage{titletoc}
\usepackage{fancyhdr}
\usepackage{titling}
\usepackage{enumitem}
\usepackage{eurosym}

\usepackage{hyperref}
\hypersetup{
    colorlinks=true,
    linkcolor=blue,
    urlcolor=blue,
    filecolor=blue}
    
\geometry{hmargin=2.cm, vmargin = 2.cm}

\usepackage{diagbox} % tableaux à double entrée

\title{Algorithmic game theories : Solutions Concept}
\author{Eliel Dzik}

\begin{document}
\date{}
\maketitle


\section{Dominated strategies}
Let consider that player 1 has $n_1$ strategies and player 2 has $n_2$ strategies.
If we show that we can prove in a polynomial time that a strategy is dominated,
 then it will be possible to find all dominated strategies in a polynomial time since we just 
 need to iterate over all strategies and $ n \times P(n) \in P(n) $ where $P(n)$ is the set of polynomial functions of $n$.

Let consider a strategy $s$ of player 1 and $g(s,t)$ the payoff of player 1 when he plays $s$ and player 2 plays $t$.
The function $t \rightarrow g(s,t)$ is convex over the set of strategies of player 2 because it is an expectancy, which convex.
This is why we just need to evaluate pure strategies of player 2.
Then, we just need to evaluate if $\forall t $pure strategy, $\exists s'$ pure such that $ g(s,t) < g(s',t)$.
If it is the case, then $s$ is dominated. This algorithm is in $O(n_1 \times n_2)$.
Finaly, the global algorithm is in $O(n_1 \times n_2 \times n_1 ) = O(n_1^2 \times n_2)$ which is polynomial.

\section{Nash equilibrium}
The first observation is that there is a symetry between the three players. 
It is trivial that no Nash equilibrium can include pure strategies.
Then, for each player $i$, we can consider the following mixed strategy : $(p_i, 1-p_i)$.
For next steps, we will only denote a strategy by $p_i$.

For player 1, the expected payoff is $$ g_1(p_1, p_2, p_3) = p_1 \times (1-p_2) \times (1-p_3) + (1-p_1) \times p_3 $$.
The first part is when player 1 plays $a_1$, then he needs that player 2 plays $b_2$ and player 3 plays $c_2$.
The second part is when player 1 plays $a_2$ and player 3 plays $c_1$.

After some calculus, we find:
$$
g(p_1, p_2, p_3) = p_1 \times (1 + p_2p_3 - p_2 - 2p_3) + p_3
$$

If $ (1 + p_2p_3 - p_2 - 2p_3) < 0 $, the best response of player 1 is $p_1 = 0$.
If $ (1 + p_2p_3 - p_2 - 2p_3) > 0 $, the best response of player 1 is $p_1 = 1$.
These two cases are not Nash equilibria since they includes pure strategies.

If $ (1 + p_2p_3 - p_2 - 2p_3) = 0 $, the best response of player 1 is $p_1 \in [0,1]$ : player 1 is indifferent.
Let study this case.
$$
(1 + p_2p_3 - p_2 - 2p_3) = 0 \\
\Leftrightarrow p_3 = \frac{1 - p_2}{2 - p_2}
$$

By symetry, we also have:
$$
p_1 = \frac{1 - p_3}{2 - p_3} \\
p_2 = \frac{1 - p_1}{2 - p_1}
$$

We develop the first equation thanks to the two others:
$$
p_3 = \frac{1 - \frac{1-p_1}{2 - \frac{1 -p_1}{2 - p_}}}{2 - \frac{1-p_1}{2 - \frac{1 -p_1}{2 - p_}}} 
$$